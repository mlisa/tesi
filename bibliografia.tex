\begin{thebibliography}{99}
\addcontentsline{toc}{chapter}{Bibliografia}

\bibitem{bib01} Bing Liu,
{\em Sentiment Analysis and Subjectivity},
in "\textit{Handbook of Natural Language Processing}", Indurkhya and Damerau, 2010.

\bibitem{bib02} Ryan Kelly,
{\em Twitter Study – August 2009},
Pear Analytics, 2009.
%https://web.archive.org/web/20110715062407/www.pearanalytics.com/blog/wp-content/uploads/2010/05/Twitter-Study-August-2009.pdf
 
\bibitem{bib03} Alec Go, Lei Huang, Richa Bhayani,
{\em Twitter Sentiment Analysis },
Entropy, 2009.

\bibitem{bib04} Alexander Pak, Patrick Paroubek,
{\em Twitter as a Corpus for Sentiment Analysis and Opinion Mining},
LREc, 2010.

\bibitem{bib05} Efthymios Kouloumpis, Theresa Wilson, Johanna Moore,
{\em Twitter Sentiment Analysis - The Good the Bad and the OMG!},
Icwsm, 2011. 

\bibitem{bib06} Apoorv Agarwal, Boyi Xie, Ilia Vovsha, Owen Rambow, Rebecca Passoneau,
{\em Sentiment Analysis of Twitter Data},
Proceedings of the Workshop on Language in Social Media, 2011.

\bibitem{bib07} Andranik Tumasjan, Timm O. Sprenger, Philipp G. Sandner, Isabell M. Welpe,
{\em Predicting Elections with Twitter: What 140 Characters Reveal about Political Sentiment},
ICWSM, 2010.

\bibitem{bib08} Hao Wang, Dogan Can, Abe Kazemzadeh, François Bar,  Shrikanth Narayanan,
{\em A System for Real-Time Twitter Sentiment Analysis of 2012 U.S. Presidential Election Cycle},
Proceedings of the 50th Annual Meeting of the Association for Computational Linguistics, 2012.

\bibitem{bib09} Ian H. Witten,
{\em Text mining},
Practical handbook of Internet computing, 2005
\end{thebibliography}